\chapter{Daten}
Im nun folgenden Kapitel werden die im Laufe dieser Arbeit und den dabei durchgeführten Experimenten verwendete Trainings- und Validierungsdaten etwas genauer eingegangen. Es wird erläutert von wo die Daten stammen, welchen Zweck sie erfüllen und wie die einzelnen Datensätze aufgebaut sind.

\section{Art der Datensätze}
Die Daten, welche in dieser Arbeit verwendet wurden können in zwei Klassen unterteilt werden:

\begin{itemize}
	\item Die Daten, welche Annotationen mit den entsprechenden Sentiments bereits mitlieferern, auch \emph{Supervised} genannt.
	\item Die zweite Klasse beinhaltet Daten, welche keine Annotationen mitliefern. Bei diesen lässt sich der Sentiment der einzelen Datensätze über Eigenschaften des Textes ableiten. Als Beispiel kann hier zum Beispiel die Emoji-Annotation aus \cite{DeriuMasterThesis} erwähnt werden. Dabei wurde der Sentiment eines Tweets daraus abgeleitet ob positive oder negative Emojis im Tweet vorhanden sind. Diese Art von Daten wird \emph{Unsupervised} genannt.
\end{itemize}

\section{Supervised}

\subsection{SemEval Tweets}
\blindtext
\subsection{MPQ News}
\blindtext
\section{Unsupervised}
\subsection{Amazon Reviews}
\blindtext
\subsection{News Corpus}
\blindtext
\section{Statistiken zu Datensätzen}
\blindtext