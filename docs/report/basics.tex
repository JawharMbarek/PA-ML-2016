\chapter{Grundlagen}
In diesem Kapitel werden die theoretischen Grundlagen, welche in den  folgenden Kapitel dieser Arbeit Verwendung finden, erläutert. Als erstes erfolgt eine kurze Einführung in maschinelles Lernen mit Neuronalen Netzen und Deep Learning. Danach wird eine spezielle Form dieser Netze vorgestellt, das sogenannte \emph{Deep Convolutional Neural Network}, welches im Rahmen dieser Arbeit verwendet wurde. Zuletzt wird noch kurz auf das sogenannte \emph{Distant Learning} \cite{deriu2016sentiment} eingegangen und dieses in Kontext mit der gegebenen Fragestellung (vgl. Kapitel 1) gesetzt.

\section{Neuronale Netze}
\blindtext
\section{Deep Learning}
\blindtext
\section{Convolutional Neural Network}
\blindtext
\section{Word Embeddings}
\blindtext
\section{Distant-Phase Learning}
\blindtext
\section{Technischer Aufbau}
Im folgenden Abschnitt wird der technische Aufbau, welcher verwendet wurde, um die in \ref{experiments} beschriebenen Experimente durchzuführen.
\subsection{Software}
\subsubsection{Vorarbeiten}
\label{technichal_setup:prework}
Der Grundaufbau der verwendeten Software wurde von Jan Deriu mithilfe von Keras\footnote{https://keras.io/} implementiert und zur Durchführung dieser Arbeit zur Verfügung gestellt. Im Rahmen dieses Grundaufbaus wurde die folgende Funktionalität bereits implementiert:

\begin{itemize}
	\item Implementation des CNN in Keras
	\item Implementation von Evaluations-Metriken
	\item Scripte mit den folgenden Funktionalitäten:
	\begin{itemize}
		\item Trainieren des CNN
		\item Laden von TSV-Dateien
		\item Vorverarbeiten von Word-Embeddings
	\end{itemize}
\end{itemize}
\subsubsection{Anforderungen}
\label{technical_setup:requirements}
Ein System, welches die in \ref{experiments} beschriebenen Experimente durchzuführen, soll folgende Eigenschaften aufweisen:

\begin{itemize}
	\item \textbf{Parametrisierbarkeit}: Dadurch dass eine grosse Anzahl kleiner Experimente durchgeführt werden muss soll das System die Möglichkeit bitten Experimente parametrisiert durchzuführen.
	\item \textbf{Wiederholbarkeit}: Experimente sollen wenn nötig mehrfach durchgeführt werden ohne einen Mehraufwand zu verursachen. 
	\item \textbf{Übersichtlichkeit}: Resultate der Experimente sollen übersichtlich und einfach zugänglich sein.
	\item \textbf{Auswertbarkeit}: Resultate sollen \fixme{Bessers Wort für Einfach?} einfach ausgewertet werden können.
\end{itemize}

Die in \ref{technichal_setup:prework} beschriebenen Vorarbeiten bitten eine Basis um damit ein System aufzubauen, welche die oben beschriebenen Eigenschaften aufweist. Allerdings werden diese Eigenschaften von dem von Jan Deriu implementierten Grundaufbau nicht abgedeckt.
\subsubsection{Funktionalität}
Um ein System, welches die im vorhergehenden Kapitel beschriebenen Eigenschaften aufweist, zu erhalten, wurden die folgenden Komponenten implementiert:

\begin{itemize}
	\item Blub
\end{itemize}
\subsubsection{Weboberfläche}
Um die dritte Anforderung nach Übersichtlichkeit zu erfüllen wurde eine Weboberfläche umgesetzt, mit welchem die Parameter und Resultate aller durchgeführten Experimente übersichtlich und an einem Ort zur Verfügung zu stellen.
\subsection{Hardware}
\blindtext