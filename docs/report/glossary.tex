% Glossar
\newglossaryentry{alkkalk}
{
  name=AlkKalk,
  description={Name unserer Anwendung. Setzt sich aus
			\underline{Alk}ohol \underline{Kalk}ulator zusammen}
}

\newglossaryentry{anwender}
{
  name=Anwender,
  description={Ist der Akteur, der AlkKalk auf seinem Android Smartphone benutzt.}
}

\newglossaryentry{applikation}
{
  name=Applikation,
  description={Anwendung}
}

\newglossaryentry{brainstorming}
{
  name=Brainstorming,
  description={Methode zur Ideenfindung, jede Idee wird
akzeptiert. Erst später werden sie gegeneinander abgewogen.}
}

\newglossaryentry{blutalkoholkonzentration}
{
  name=Blutalkoholkonzentration,
  description={Die Konzentration des Ethanols pro Liter Blut,
    wird üblicherweise in Promille angegeben.}
}

\newglossaryentry{materialdesign}
{
  name=Material Design,
  description={Material Design ist eine vom Unternehmen Google entwickelte Designsprache,
    die in Android verwendet wird.}
}

\newglossaryentry{metabolismus}
{
  name=Metabolismus,
  description={Stoffwechsel}
}

\newglossaryentry{persistent}
{
  name=Persistent,
  description={Persistent heisst, dass die Daten auch über die Laufzeit
    eines Programms hinaus gespeichert werden.}
}

\newglossaryentry{spezifikation}
{
  name=Spezifikation,
  description={Beschreibung des Produkts durch Auflistung seiner Anforderungen}
}

\newglossaryentry{userinterface}
{
  name=User Interface,
  description={Englisch für Benutzerschnittstelle,
    über welche der Anwender die Applikation steuern kann.}
}

\newglossaryentry{usability}
{
  name=Usability,
  description={Englisch für Benutzerfreundlichkeit}
}

\newglossaryentry{playstore}
{
  name=Google Play Store,
  description={Der Google Play Store ist ein App Store, der meist auf
    Smartphones und Tablet-Computern mit dem Betriebssystem Android ausgeliefert wird}
}

\newglossaryentry{getrank}
{
  name=Getränk,
  description={Ein Getränk ist ein Sammelbegriff für zum trinken zubereitete alkoholische Getränke. Alkoholische Getränke oder alkoholhaltige Getränke, auch Alkoholika oder (vor allem in Bezug auf Spirituosen) geistige Getränke genannt, sind Getränke, die Trinkalkohol (Ethanol) enthalten, der in Lebensmitteln meist nur als Alkohol bezeichnet wird.}
}


\newglossaryentry{kohasion}
{
  name=Kohäsion,
  description={Sollte möglichst stark sein und steht dafür,
    wie gut logische Einheiten durch \underline{eine} Programmeinheit umgesetzt wird}
}

\newglossaryentry{kopplung}
{
  name=Kopplung,
  description={Beschreibt das Mass der Verknüpfung zwischen Softwaremodulen.
    Das Ziel ist es, diese Abhängigkeit möglichst gering zu halten.}
}

\newglossaryentry{Wearables}
{
  name=Wearables,
  description={Wearables sind Tragbare Computersysteme, ein Beispiel hierfür wäre die IWatch von Apple.}
}



